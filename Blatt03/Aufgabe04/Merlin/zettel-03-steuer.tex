\documentclass[a4paper,10pt]{article}

% Hier die Nummer des Blatts und Autoren angeben.
\newcommand{\blatt}{3}
\newcommand{\autor}{Merlin Steuer}

\usepackage{hci}

\usepackage{fancyvrb}
\usepackage{icomma}

\newcommand{\dunderline}[1]{\underline{\underline{#1}}}

\begin{document}
% Seitenkopf mit Informationen
\kopf
\renewcommand{\figurename}{Figure}

\aufgabe{4}

Beide Varianten finden unter Windows 7 statt.

\begin{itemize}
\item Zunächst betrachten wir den Best-Case. Der Browser Firefox befindet sich im Start-Menü unter den Oft verwendeten Programmen ganz oben. Er startet binnen 2 Sekunden. Die Startseite ist bereits Google, standartmäßig ist das Eingabefeld von Google schon ausgewählt, der Nutzer kann also sofort los tippen. Außerdem goooglet er oft nach Interaktionsdesign, so dass der Suchbegriff bereits nach Eingabe des Buchstaben i erscheint, daraufhin drückt der Nutzer nur noch auf die Pfeil-Unten-Taste und Enter, um die Suche zu starten. Die Hände befinden sich bei Beginn bereits auf der Tastatur, der Benutzer ist außerdem sehr erfahren im Umgang mit Computern.

\begin{Verbatim}[commandchars=\\\{\}]
\textbf{GOAL: GOOGLE-FOR-IXD}
    PRESS-WINDOWS-KEY
    PRESS-DOWN-KEY
    PRESS-ENTER
    WAIT-FOR-FIREFOX
    TYPE-LETTER-I
    PRESS-DOWN-KEY
    PRESS-ENTER
\end{Verbatim}

Zeitliche Analyse:
\begin{center}
    \begin{tabular}{|l|c|}
    \hline
    Aktion & Zeit \\
    \hline
    K & $0,28s$ \\
    K & $0,28s$ \\
    K & $0,28s$ \\
    $W(2s)$ & $2s$ \\
    K & $0,28s$ \\
    K & $0,28s$ \\
    K & $0,28s$ \\
    \hline
    \hline
    \textbf{Gesamt} & $3,68s$ \\
    \hline
    \end{tabular}
\end{center}

\item Der Benutzer ist unerfahren im Umgang mit Computern und kennt keine Shortcuts und weiß mit der Autovervollständigung von Google nichts anzufangen.

\begin{Verbatim}[commandchars=\\\{\}]
\textbf{GOAL: GOOGLE-FOR-IXD}
    LOCATE-START-BUTTON
    MOVE-MOUSE-TO-START-BUTTON
    CLICK-START-BUTTON
    LOCATE-ENTRY "Alle Programme"
    MOVE-TO-ENTRY "Alle Programme"
    CLICK-LEFT
    LOCATE-ENTRY "Mozilla"
    MOVE-TO-ENTRY "Mozilla"
    CLICK-LEFT
    LOCATE-ENTRY "Firefox"
    MOVE-TO-ENTRY "Firefox"
    CLICK-LEFT
    WAIT-FOR-FIREFOX
    LOCATE-ADDRESS-BAR
    MOVE-TO-ADDRESS-BAR
    CLICK-LEFT % Der Ganze Text soll bereits markiert sein
    TYPE-IN "www.google.de"
    PRESS-ENTER
    WAIT-FOR-PAGE % 0.5s
    LOCATE-SEARCH-FIELD
    MOVE-TO-SEARCH-FIELD
    CLICK-LEFT
    TYPE-IN "Interaktionsdesign"
    LOCATE-SEARCH-BUTTON
    MOVE-TO-SEARCH-BUTTON
    CLICK-LEFT % Geschafft!
\end{Verbatim}

Zeitliche Analyse:
\begin{center}
    \begin{tabular}{|l|c|}
    \hline
    Aktion & Zeit \\
    \hline
    M       & $1,2s$ \\
    P       & $1,1s$ \\
    BB      & $0,2s$ \\
    M       & $1,2s$ \\
    P       & $1,1s$ \\
    BB      & $0,2s$ \\
    M       & $1,2s$ \\
    P       & $1,1s$ \\
    BB      & $0,2s$ \\
    M       & $1,2s$ \\
    P       & $1,1s$ \\
    BB      & $0,2s$ \\
    $W(2s)$ & $2,0s$ \\
    M       & $1,2s$ \\
    P       & $1,1s$ \\
    BB      & $0,2s$ \\
    H       & $0,4s$ \\
    $T(13)$ & $13 \cdot 0,28s = 3,64s$ \\
    K       & $0,28s$ \\
    $W(0,5s)$ & $0,5s$ \\
    H       & $0,4s$ \\
    M       & $1,2s$ \\
    P       & $1,1s$ \\
    BB      & $0,2s$ \\
    H       & $0,4s$ \\
    $T(19)$ & $19 \cdot 0,28s = 5,32s $ \\
    H       & $0,4s$ \\
    M       & $1,2s$ \\
    P       & $1,1s$ \\
    BB      & $0,2s$ \\
    \hline
    \hline
    \textbf{Gesamt} & $30,84s$ \\
    \hline
    \end{tabular}
\end{center}

Der erfahrene Nutzer benötigt also nur $\left( \frac{3,64s}{30,84s} \right) \cdot 100 \approx \dunderline{12 \%}$ der Zeit, die der unerfahrene Benutzer benötigt\footnote{\glqq So much pain.\grqq}, ist also ca. zehnfach schneller.

\end{itemize}

\end{document}
