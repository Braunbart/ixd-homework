\documentclass[a4paper,10pt]{article}

% Hier die Nummer des Blatts und Autoren angeben.
\newcommand{\blatt}{2}
\newcommand{\autor}{Merlin Steuer, Till Schander, Lennart Bergmann}

\usepackage{hci}

\begin{document}
% Seitenkopf mit Informationen
\kopf
\renewcommand{\figurename}{Figure}

\aufgabe{3}
Im folgenden werden verschiedene Aufteilungen der Menüpunkte innerhalb eines Menüs in Hinsicht auf ihre Benutzerfreundlichkeit nach Hick betrachtet.

In den Tabellen bezeichnet ein x einen Menü-Eintrag, welcher vom Benutzer geklickt werden kann sowie ein o ein Element in einem Menü, welches eine Eintrags-Gruppe beschreibt. Ein o gruppiert alle links unter ihm und unter ihm stehenden Menüeinträge bis zum nächsten o.


@TODO: Prüfen, ob alles richtig ist und nichts vergessen wurde \\
@TODO: was wurde warum weggelassen \\
@TODO: schreiben, dass Berechnungen jeweils worst case sind \\
@TODO: müsste es nicht in den Ergebnissen jeweils $2\cdot b$ sein? In den Folien ist es nur b \\
@TODO: Fazit (Für naive und erfahrene Nutzer sind H und J am besten)\\

A)\\
\begin{tabular}{|c|c|c|c|c|c|l|l}
\cline{1-6} x & x & x & x & x & x & $1\cdot 6$ \: \: \: \: \: \: \: & Naiv: $T = b\cdot 6$ \\
\cline{1-6}   &   &   &   &   &   &   & Erfahren: $T = b\cdot \log(6+1) = b\cdot 2.81$\\
\cline{1-6}
\end{tabular} \\

B)\\
\begin{tabular}{|c|c|c|c|c|c|l|l}
\cline{1-6} x & x & x & x &   & o & $1\cdot 5$ und $1\cdot 2$ & Naiv: $T = b\cdot 5+b\cdot 2 = b\cdot 7$ \\
\cline{1-6}   &   &   &   & x & x &   & Erfahren: $T = b\cdot \log(5+1)+b\cdot \log(2+1) = b\cdot 4.17$ \\
\cline{1-6}
\end{tabular} \\

C)\\
\begin{tabular}{|c|c|c|c|c|c|l|l}
\cline{1-6} x & x & x &   &   & o & $1\cdot 4$ und $1\cdot 3$ & Naiv: $T = b\cdot 4+b\cdot 3 = b\cdot 7$ \\
\cline{1-6}   &   &   & x & x & x &   & Erfahren: $T = b\cdot \log(4+1)+b\cdot \log(3+1) = b\cdot 4.32$ \\
\cline{1-6}
\end{tabular} \\

D)\\
\begin{tabular}{|c|c|c|c|c|c|l|l}
\cline{1-6} x & x &   &   &   & o & $1\cdot 3$ und $1\cdot 4$ & Naiv: $T = b\cdot 3+b\cdot 4 = b\cdot 7$ \\
\cline{1-6}   &   & x & x & x & x &   & Erfahren: $T = b\cdot \log(3+1)+b\cdot \log(4+1) = b\cdot 4.32$ \\
\cline{1-6}
\end{tabular} \\

E)\\
\begin{tabular}{|c|c|c|c|c|c|l|l}
\cline{1-6} x &   &   &   &   & o & $1\cdot 2$ und $1\cdot 5$ & Naiv: $T = b\cdot 2+b\cdot 5 = b\cdot 7$ \\
\cline{1-6}   & x & x & x & x & x &   & Erfahren: $T = b\cdot \log(2+1)+b\cdot \log(5+1) = b\cdot 4.17$ \\
\cline{1-6}
\end{tabular} \\

F)\\
\begin{tabular}{|c|c|c|c|c|c|l|l}
\cline{1-6} x & x &   & o &   & o & $1\cdot 4$ und $1\cdot 2$ & Naiv: $T = b\cdot 4+b\cdot 2 = b\cdot 6$ \\
\cline{1-6}   &   & x & x & x & x &   & Erfahren: $T = b\cdot \log(4+1)+b\cdot \log(2+1) = b\cdot 3.91$ \\
\cline{1-6}
\end{tabular} \\

G)\\
\begin{tabular}{|c|c|c|c|c|c|l|l}
\cline{1-6} x &   & o &   &   & o & $1\cdot 3$ und $1\cdot 3$ & Naiv: $T = b\cdot 3+b\cdot 3 = b\cdot 6$ \\
\cline{1-6}   & x & x & x & x & x &   & Erfahren: $T = b\cdot \log(3+1)+b\cdot \log(3+1) = b\cdot 3.17$ \\
\cline{1-6}
\end{tabular} \\

H)\\
\begin{tabular}{|c|c|c|c|c|c|l|l}
\cline{1-6}   &   & o &   &   & o & $1\cdot 2$ und $1\cdot 3$ & Naiv: $T = b\cdot 2+b\cdot 3 = b\cdot 5$ \\
\cline{1-6} x & x & x & x & x & x &   & Erfahren: $T = b\cdot \log(2+1)+b\cdot \log(3+1) = b\cdot 2.58$ \\
\cline{1-6}
\end{tabular} \\

I)\\
\begin{tabular}{|c|c|c|c|c|c|l|l}
\cline{1-6}   & o &   &   &   & o & $1\cdot 2$ und $1\cdot 4$ & Naiv: $T = b\cdot 2+b\cdot 4 = b\cdot 6$ \\
\cline{1-6} x & x & x & x & x & x &   & Erfahren: $T = b\cdot \log(2+1)+b\cdot \log(4+1) = b\cdot 3.91$ \\
\cline{1-6}
\end{tabular} \\

J)\\
\begin{tabular}{|c|c|c|c|c|c|l|l}
\cline{1-6}   & o &   & o &   & o & $1\cdot 3$ und $1\cdot 2$ & Naiv: $T = b\cdot 3+b\cdot 2 = b\cdot 5$ \\
\cline{1-6} x & x & x & x & x & x &   & Erfahren: $T = b\cdot \log(3+1)+b\cdot \log(2+1) = b\cdot 2.58$ \\
\cline{1-6}
\end{tabular} \\

\end{document}
