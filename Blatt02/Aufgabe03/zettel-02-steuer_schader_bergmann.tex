\documentclass[a4paper,10pt]{article}

% Hier die Nummer des Blatts und Autoren angeben.
\newcommand{\blatt}{2}
\newcommand{\autor}{Merlin Steuer, Till Schander, Lennart Bergmann}

\usepackage{hci}

\begin{document}
% Seitenkopf mit Informationen
\kopf
\renewcommand{\figurename}{Figure}

\aufgabe{3}

@TODO: Prüfen, ob alles richtig ist und nichts vergessen wurde \\
@TODO: Einleitender Satz \\
@TODO: was wurde warum weggelassen \\
@TODO: was bedeutet x und o \\
@TODO: schreiben, dass Berechnungen jeweils worst case sind \\
@TODO: müsste es nicht in den Ergebnissen jeweils 2*b sein? In den Folien ist es nur b \\
@TODO: Fazit (Für naive und erfahrene Nutzer sind H und J am besten)\\

A)\\
\begin{tabular}{|c|c|c|c|c|c|l|l}
\cline{1-6} x & x & x & x & x & x & 1*6 \: \: \: \: \: \: \: & Naiv: T = b*6 \\
\cline{1-6}   &   &   &   &   &   &   & Erfahren: T = b*log(6+1) = b*2.81\\
\cline{1-6}
\end{tabular} \\

B)\\
\begin{tabular}{|c|c|c|c|c|c|l|l}
\cline{1-6} x & x & x & x &   & o & 1*5 und 1*2 & Naiv: T = b*5+b*2 = b*7 \\
\cline{1-6}   &   &   &   & x & x &   & Erfahren: T = b*log(5+1)+b*log(2+1) = b*4.17 \\
\cline{1-6}
\end{tabular} \\

C)\\
\begin{tabular}{|c|c|c|c|c|c|l|l}
\cline{1-6} x & x & x &   &   & o & 1*4 und 1*3 & Naiv: T = b*4+b*3 = b*7 \\
\cline{1-6}   &   &   & x & x & x &   & Erfahren: T = b*log(4+1)+b*log(3+1) = b*4.32 \\
\cline{1-6}
\end{tabular} \\

D)\\
\begin{tabular}{|c|c|c|c|c|c|l|l}
\cline{1-6} x & x &   &   &   & o & 1*3 und 1*4 & Naiv: T = b*3+b*4 = b*7 \\
\cline{1-6}   &   & x & x & x & x &   & Erfahren: T = b*log(3+1)+b*log(4+1) = b*4.32 \\
\cline{1-6}
\end{tabular} \\

E)\\
\begin{tabular}{|c|c|c|c|c|c|l|l}
\cline{1-6} x &   &   &   &   & o & 1*2 und 1*5 & Naiv: T = b*2+b*5 = b*7 \\
\cline{1-6}   & x & x & x & x & x &   & Erfahren: T = b*log(2+1)+b*log(5+1) = b*4.17 \\
\cline{1-6}
\end{tabular} \\

F)\\
\begin{tabular}{|c|c|c|c|c|c|l|l}
\cline{1-6} x & x &   & o &   & o & 1*4 und 1*2 & Naiv: T = b*4+b*2 = b*6 \\
\cline{1-6}   &   & x & x & x & x &   & Erfahren: T = b*log(4+1)+b*log(2+1) = b*3.91 \\
\cline{1-6}
\end{tabular} \\

G)\\
\begin{tabular}{|c|c|c|c|c|c|l|l}
\cline{1-6} x &   & o &   &   & o & 1*3 und 1*3 & Naiv: T = b*3+b*3 = b*6 \\
\cline{1-6}   & x & x & x & x & x &   & Erfahren: T = b*log(3+1)+b*log(3+1) = b*3.17 \\
\cline{1-6}
\end{tabular} \\

H)\\
\begin{tabular}{|c|c|c|c|c|c|l|l}
\cline{1-6}   &   & o &   &   & o & 1*2 und 1*3 & Naiv: T = b*2+b*3 = b*5 \\
\cline{1-6} x & x & x & x & x & x &   & Erfahren: T = b*log(2+1)+b*log(3+1) = b*2.58 \\
\cline{1-6}
\end{tabular} \\

I)\\
\begin{tabular}{|c|c|c|c|c|c|l|l}
\cline{1-6}   & o &   &   &   & o & 1*2 und 1*4 & Naiv: T = b*2+b*4 = b*6 \\
\cline{1-6} x & x & x & x & x & x &   & Erfahren: T = b*log(2+1)+b*log(4+1) = b*3.91 \\
\cline{1-6}
\end{tabular} \\

J)\\
\begin{tabular}{|c|c|c|c|c|c|l|l}
\cline{1-6}   & o &   & o &   & o & 1*3 und 1*2 & Naiv: T = b*3+b*2 = b*5 \\
\cline{1-6} x & x & x & x & x & x &   & Erfahren: T = b*log(3+1)+b*log(2+1) = b*2.58 \\
\cline{1-6}
\end{tabular} \\

\end{document}
