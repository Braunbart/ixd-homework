\documentclass[a4paper,10pt]{article}

% Hier die Nummer des Blatts und Autoren angeben.
\newcommand{\blatt}{2}
\newcommand{\autor}{Merlin Steuer}

\usepackage{hci}

\begin{document}
% Seitenkopf mit Informationen
\kopf
\renewcommand{\figurename}{Figure}

\aufgabe{2}
Die Autoren möchte in ihrem Artikel eine Möglichkeit beschreiben, Verhalten und Reaktionszeiten von Menschen in einem Modell darzustellen. Hierzu erläutern sie zunächst das auch schon in der Vorlesung angesprochene Modell des \textit{Model Human Processor}s. Besprochen wird unter anderem die Rolle der Forschung in der Psychologie über diese Themen.

Weiterhin zeigen die Autoren an einigen Beispielen, wie die sog. \textit{Cycle Time} der einzelnen Prozessoren des Menschen sich auf bestimmte Lebenssituationen auswirkt. So wird Beispielhaft vorgerechnet, wieviele Bilder pro Sekunde in einem Film gezeigt werden müssen, um die Illusion einer Bewegung vorzutäuschen oder wie lange es dauert, einen Knopf in einer bestimmten Entfernung mit der Hand zu drücken. Unterschieden wird dabei grundsätzlich zwischen 3 verschiedenen Geschwindigkeitsstufen, dem \textit{Slowman}, dem \textit{Middleman} sowie dem \textit{Fastman}.

\end{document}
