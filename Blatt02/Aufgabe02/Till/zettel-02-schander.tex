\documentclass[a4paper,10pt]{article}

% Hier die Nummer des Blatts und Autoren angeben.
\newcommand{\blatt}{2}
\newcommand{\autor}{Till Schander (6682565)}

\usepackage{hci}

\begin{document}
% Seitenkopf mit Informationen
\kopf
\renewcommand{\figurename}{Abbildung}

\aufgabe{2}
Der Model Human Processor wurde entwickelt, da es in der Psychologie zwar viele Studien und empirische Untersuchungen zu den kognitiven Fähigkeiten von Menschen gab, diese aber nicht sehr zugänglich waren. Personen aus anderen Disziplinen konnten kaum wissen über diese Fähigkeiten in ihre Arbeit einfließen lassen. Eine dieser Disziplinen ist die der Entwicklung von Interfaces.

Interface Designer können mit dem Model Human Processor relativ leicht herausfinden, wie lange ein Nutzer für eine bestimmte Aktion benötigt. Die so gewonnenen Erkenntnisse sind zwar nur approximativ, aber in den meisten Fällen gut genug.

Der Model Human Processor besteht aus Prozessoren, Speichern und den Verbindungen zwischen dieses. Dazu gibt es noch eine Liste von Operationsprinzipien. Dazu kommen noch Parameter, mit denen Eigenschaften der Prozessoren und Speicher festgelegt werden können.

Die Prozessoren sind perzeptuell, kognitive- und motorisch. Für einen Zyklus in einem der Prozessoren wird eine Dauer von 100ms veranschlagt. Je nach dem ob eine Aufgabe seriell, oder parallel ausgeführt werden kann, kann mit diesen Zeiten die Gesamtdauer der Aufgabe berechnet werden.

Die Speicher in dem Modell sind das Arbeits- und das Langzeitgedächtnis, sowie zwei Unterbereiche des Arbeitsgedächtnisses der visuelle und der auditive Bildspeicher.

Der zentrale Arbeitsschritt des Model Human Processors ist der sogenannte Recognize-Act cycle des kognitiven Prozessors. Hier werden in jedem Zyklus zentrale Aufgaben ausgeführt. Dies kann bspw. das vergleichen von Informationen im Arbeitsgedächtnis sein, oder das Entscheiden für eine Bewegung, welche dann vom motorischen Prozessor durchgeführt wird.

\end{document}
